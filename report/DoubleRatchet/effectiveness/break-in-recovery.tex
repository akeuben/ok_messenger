In order to provide break-in recovery, Double Ratchet introduces two new ratchets, the Diffie-Hellman ratchet 
and the root key chain. These systems are in place to occationally recreate the sending and receiving ratchets 
so even if an attacker manages to steal one key, once the sending/receiving ratchet is reset they won't be able 
to derive future keys, as the next sending/receiving key is not related to the key they have stolen. We will explore this in more detail:

\textbf{The DH Ratchet}

The Diffie-Hellman ratchet is built upon the Diffie-Hellman algorithm. Bob and alice continously generate new 
public/private key pairs. They send their public key to the other party, and upon reception of they key, they can derive 
a new shared secret. This secret is used in the root key chain to generate the keys for the sending/receiving ratchets.

Suppose bob wants to establish a new sending/receiving chain. He generates 
a new Public/Private key pair. He sends his public key to Alice. Alice does the same, and also generates a new public/private 
key pair. Alice sends her new public key to bob. They can then both use Diffie-Hellman on their private key and the other party's public key 
to generate a shared secret. This shared secret is used to generate Alice's sending chain key and bob's receiving chain key. This process 
repeates as long as Alice and Bob keep sending messages.

\textbf{The Root Key Chain}

The output from Alice's DH calculation above becomes the input (along with the root key chain) to generate Alice's 
sending ratchet key (resetting the ratchet). Then, Bob will compute the same shared secret, and this will be the input (along with the root key chain)
to generate Bob's receiving chain. As the root key chain is in sync between the two parties (initialized with the same shared secret),
then we have established the desired property of Alice's sending chain being equal to Bob's receiving chain, and vice-versa.
We can then ratchet the sending and receiving ratchet as normal to generate keys for individual encryptions and decryptions.
