Double ratchet is the core key generation algorithm used within the signal protocol.
The idea is that every message is sent with a new key, so even if an attacker manages 
to steal an encryption/decryption key and read one message, they won't be able to read 
any other messages. 

The core idea is that Alice and Bob have two core ratchets each, a sending ratchet and a 
receving ratchet. If alice sends a message to bob, she will perform a ratchet step on her sending 
ratchet to get a key $k$ to encrypt the message with. Bob, upon reception of the message will perform 
a ratchet step on his receving ratchet to get the same key $k$. The core idea here is that each party's 
sending ratchet is linked to the other party's receiving ratchet, and vice-versa.

In this section, we will discuss the implementation and efficiveness of 
the Double Ratchet algorithm, including how desired security properties of double ratchet are 
achieved by expanding upon the two ratchets as described above.
